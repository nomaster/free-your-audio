\documentclass[aspectratio=1610]{beamer}

\usepackage[ngerman]{babel}
\usepackage[utf8]{inputenc}
\usepackage[absolute,overlay]{textpos}
\usepackage{xcolor,ulem}

\definecolor{black}{HTML}{3C3C3C}
\definecolor{green}{HTML}{BBE08C}

\mode<presentation>
{
  \usetheme[compress]{Torino}
}

\newcommand{\src}[1]{
  \raggedright{
    \footnote{
      \small{\url{#1}}
    }
  }
}

\newcommand{\img}[1]{
  \includegraphics[width=\columnwidth]{#1}
}

\newcommand{\two}[2]{
  \begin{columns}[1]
    \column{.5\textwidth}
    #1
    \column{.5\textwidth}
    #2
  \end{columns}
}

\newcommand{\concept}[2]{
  \footnotesize{
    \begin{block}{#1}
      \pause
      #2
    \end{block}
  }
}

\newcommand{\pattern}[2]{
  \begin{alertblock}{Problem}
    #1
  \end{alertblock}
  \pause
  \begin{exampleblock}{Lösung}
    #2
  \end{exampleblock}
}

\newcommand{\boxx}[1]{
  \colorbox{black}{
    \color{green}{#1}
  }
}

\newcommand{\imgby}[3]{
  \begin{textblock*}{\paperwidth}[0.15,0](\textwidth,\textheight)
    \raggedright{
      \boxx{
        \tiny{
          Abbildung von \href{#2}{#1} (#3)
        }
      }
    }
  \end{textblock*}
}

\AtBeginSubsection[]{
  \begin{frame}
    \tableofcontents[currentsection, currentsubsection]
  \end{frame}
}


\title{Free Your Audio}

\author[Mic]{Mic \flq nomaster@chaosdorf.de\frq}

\institute[chaosdorf]{Chaos Computer Club Düsseldorf / Chaosdorf e.V.}

\date[]{29. August 2012}

\begin{document}

  \begin{frame}
    \titlepage
  \end{frame}

  \begin{frame}{Übersicht}
    \tableofcontents
  \end{frame}

  \section{Motivation}

  \begin{frame}{Warum dieser Vortrag?}
    \begin{itemize}
      \item Compact Disk ist voll achtziger
      \pause
      \item FTP zugemüllt
      \pause
      \item Mir fallen die Ohren ab
      \pause
      \item Wertschätzung » Faven
      \pause
      \item Freie Tools sind besser
    \end{itemize}
  \end{frame}

  \section{Datenträger}

  \begin{frame}{Compact Disk}
    \two{
      \begin{enumerate}
        \renewcommand{\theenumi}{\Alph{enumi}}
        \item Polykarbonat mit “Lands“ und “Pits” (Daten leben hier)
        \item Spiegelnde Schicht
        \item Schutzschicht
        \item Beflockte Schicht mit Artwork
        \item Laser-Kopf schießt durch Polykarbonat und nimmt das reflektierte Licht auf
      \end{enumerate}
    }{
      \img{cd-layers.pdf}
      \imgby{Wikimedia Commons}{https://en.wikipedia.org/wiki/File:CD\_layers.svg}{CC-BY-SA}
    }
  \end{frame}

  \begin{frame}{Abspielen vs. Kopieren}
    \begin{itemize}
      \item Daten kopieren $\rightarrow$ Informationen übertragen
      \item Kopierschutz $\rightarrow$ Schutz vor Informationsübertragung (?!)
      \item Archivieren $\rightarrow$ sicheres Kopieren
    \end{itemize}
  \end{frame}

  \begin{frame}{Archivieren}
    \begin{itemize}
      \item Für die Zukunft (lebendige Daten)
      \item Verlustfrei (\textsl{alle} Daten)
      \item Zugänglich (offene Formate)
      \item Heute: dezentral (Replikation)
    \end{itemize}
  \end{frame}

  \begin{frame}{Quelle}
    Audio CD (Red Book)
    \begin{itemize}
      \item 1980 von Philips und Sony herausgegeben
      \item Kein freies Format (bezahlt für Lizenzen!)
      \item 2-channel signed 16-bit Linear PCM sampled at 44,100 Hz
      \item \sout{Kopier}Abspielschutz lässt uns hassen
    \end{itemize}
  \end{frame}

  \begin{frame}{Step 1: Rippen}
    \concept{Paranoia}{
      CDparanoia liest korrekt und zuverlässig Audiodaten von billigen,
      fehlerhaften Laufwerken aus. Außerdem liest es defekte CDs und repariert
      den Datenstrom wenn möglich.
      \src{https://www.xiph.org/paranoia/faq.html}
    }
    \begin{itemize}
      \pause
      \item Kopiert Daten digital von der CD
      \pause
      \item Stellt sicher, dass dabei nichts schief geht
      \pause
      \item Kompensiert “Herstellungsfehler” weitesgehend
    \end{itemize}
    \pause
    “Suspicion breeds confidence!” (Brazil)
  \end{frame}


  \begin{frame}{Step 2: Komprimieren}
    \concept{Free Lossless Audio Codec}{
      FLAC ist der schnellste und am weitesten verbreitete Codec für
      verlustfreie Audiokomprimierung; und der einizge, der nicht-propietär und
      ohne Patentbedrohung ist. Er hat eine quelloffene Referenzimplementierung,
      ein gut dokumentiertes Format und API, sowie mehrere unabhängige
      Implementierungen.\src{http://flac.sourceforge.net/}
    }
    \begin{itemize}
      \pause
      \item Komprimiert Audiodaten verlustfrei
      \pause
      \item Erstellt Dateien in offenem Format
      \pause
      \item Ermöglicht Speichern von Metadaten
    \end{itemize}
  \end{frame}

  \begin{frame}{Step 3: Indizieren}
    \concept{MusicBrainz Picard}{
      Picard ist der offizielle, plattformunabhängige Tagger des
      MusicBrainz-Projekts. Er unterstützt die Mehrheit an Audio-Dateiformaten
      und kann Audio-Fingerprints sowie CD-Erkennung anwenden.
      \src{http://musicbrainz.org/doc/MusicBrainz\_Picard}
    }
    \begin{itemize}
      \pause
      \item Offene Datenbank für Metadaten
      \pause
      \item Schreibt Metadaten in die Datei-Header
      \pause
      \item Kann Werke “erkennen”
    \end{itemize}
  \end{frame}

  \begin{frame}{Step 4: Konsumieren}
    \concept{Music Player Daemon}{
      MPD ist ein flexibler, leistungsstarker Dienst zum Abspielen von Musik.
      Mit Plug-Ins und Bibliotheken unterstützt er eine Vielzahl von
      Dateiformaten und kann durch sein Netzwerkprotokoll gesteuert werden.
      \src{http://mpd.wikia.com/wiki/Music\_Player\_Daemon\_Wiki}
    }
    \begin{itemize}
      \pause
      \item Erstellt Datenbank von Metadaten
      \pause
      \item Macht Inhalte auffindbar
      \pause
      \item Spielt Dateien ab
    \end{itemize}
  \end{frame}

  \begin{frame}{Step 5: Transkodieren (Format)}
    \concept{Ogg Vorbis}{
      Ogg Vorbis ist eine vollständig offene, patentfreie, professionelle Lösung
      für Audio-Kodierung und -Streaming mit allen Vorteilen quelloffener
      Software.
      \src{http://vorbis.com/}
    }
    \begin{itemize}
      \pause
      \item Komprimiert Audiodateien verlustbehaftet
      \pause
      \item Erstellt Dateien in ebenso offenem Format
      \pause
      \item Metadaten werden ebenso gespeichert
    \end{itemize}
  \end{frame}

  \begin{frame}{Step 5: Transkodieren (Skript)}
    \concept{Flac2All}{
      Flac2All ist ein multi-thread Skrtipt, welches deine Sammlung an
      FLAC-Dateien in Ogg Vorbis, MP3 oder FLAC umwandelt und dabei die
      Datei-/Verzeichnisstruktur übernimmt.
      \src{http://freecode.com/projects/flac2all}
    }
    \begin{itemize}
      \pause
      \item Transkodiert von FLAC nach Vorbis
      \pause
      \item Überspringt vorhandene Dateien
      \pause
      \item Kopiert Metadaten
    \end{itemize}
  \end{frame}

  \begin{frame}{Step 6: Auf die Straße tragen}
    \concept{Rockbox}{
      Rockbox ist ein freier Firmware-Ersatz für digitale Musik-Player.
      \src{http://www.rockbox.org/}
    }
    \begin{itemize}
      \pause
      \item Leistungsfähiger als deine Herstellerfirmware
      \pause
      \item Kann freie Formate tatsächlich abspielen
      \pause
      \item Audiophil (Normalisierung, gapless Playback, graphischer Equalizer)
    \end{itemize}
  \end{frame}

\end{document}
